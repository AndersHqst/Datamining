%\section{Background}
%\label{sec:background}
%In this section we present the problem statement, which is based on some interest expressed by Statistico, combined with questions we find relevant to try and answer with different data mining techniques. We also provide a short introduction to technologies used by Statistico and discuss how they can, and cannot, be used to solve the problem at hand.

\subsection{Problem statement}
\label{subsec:problem_statement}
Based on Statistico's business model and data introduced in secton \ref{sec:introduction}, we want to find out how well we can extract basic statistics from website \texttt{HTML}. We want to be able to give an overview of the distribution of content management systems (\texttt{CMS}), server software, and search engin optimization \texttt{SEO} related figures.

In addition, we want to investigate if it is possible to derive interesting correlations or patterns in the meta-data. We will use data mining software to experiment with different clustering algorithms, and see if we can derive any meaningful clusters in the data.

A very important measure for any website is its \texttt{PageRank}\footnote{PageRank is a qualitative grade given to websites by Google, affecting a sites exposure on their search engine.}. We want to investigate if it is possible to predict these values based on a relatively small number of features, compared to the complex feature set used by Google to calculate the \texttt{PageRank}. It seems unlikely, or at least not intuitive, that we should be able to derive these numbers from a data mining process that is essentially a simplification of what Google does, but non the less, the results could be interesting.

%\subsection{Meta data}
%\label{subsec:meta_data}
%We need to decide on a set of attributes that we want to collect from each website. In section \ref{sec:background} we will give an overview of these. Besides the data that we can get directly from the HTML of the websites, we can use Alexa\footnote{By Alexa we refer to Amazon's Alexa Web Information Service, which displays various data figures on websites around the world. http://aws.amazon.com/awis/} to enrich our data set with important figures such as back-link and page load time, which would be hard to get ourselves, but is easy to retrieve as a service.

\subsection{Attributes}
\label{subsec:attributes}

The complete list of attributes in the data set can be found in table \ref{tab:all_attributes} in appendix \ref{apx:attributes}. Note that we have enriched the data set, with website meta-data and statistics from Amazon's Alexa\cite{alexa} service. These attributes carry an \texttt{alexa} prefix.

 Attributes with the \texttt{has\_content} prefix, indicate whether a website contains content pertaining to a certain category such as {\it technology} or {\it sports}. The \texttt{has\_js} attributes indicate whether a given JavaScript libraries is used by a website. For a description of the remaining attributes, please refer to table \ref{tab:all_attributes}.

In the preprocessed data, we found that certain attributes were almost static across the whole dataset and decided to exclude these from the actual mining process, as these provided no further insight about the websites. This selection process was performed manually, and reduced the final data set to 29 attributes, which gives 70325 values with 1874, or 2.55\%, missing.

Further, the reduced data set was grouped into three logical subsets, each accounting for some specific aspect of the websites. These final subsets can be seen in table \ref{tab:cluster_attributes}. These subsets were used for the actual analysis.

\htab{l l}
{
\toprule
Subset & Attributes in subset\\
\midrule
Rank subset & \texttt{alexa\_load\_time}, \texttt{alexa\_rank}, \texttt{alexa\_rank\_dk}, \\
& \texttt{alexa\_links\_in}, \texttt{internal\_links\_count}, \\
& \texttt{external\_links\_count}, \texttt{page\_rank}, \texttt{title\_tag}, \\
& \texttt{has\_description}, \texttt{has\_keywords}, \texttt{img\_count} \\
\midrule
Technology subset & \texttt{html5}, \texttt{html5\_tags}, \texttt{has\_js\_jquery}, \\
& \texttt{server}, \texttt{cms}, \texttt{has\_analytics} \\
\midrule
Content subset & \texttt{has\_content\_business}, \texttt{has\_content\_film}, \\
& \texttt{has\_content\_food}, \texttt{has\_content\_games}, \\
& \texttt{has\_content\_health}, \texttt{has\_content\_music}, \\
& \texttt{has\_content\_news}, \texttt{has\_content\_shop}, \\
& \texttt{has\_content\_sport}, \texttt{has\_content\_technology}, \\
& \texttt{has\_content\_transport}, \texttt{has\_content\_xxx} \\
\bottomrule
}{The different subsets used for in clustering process.}{tab:cluster_attributes}