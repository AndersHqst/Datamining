\section{Solution}
\label{sec:solution}

In this section we give a technical overview of our solution. The {Python} programming language is used throughout the solution and, unless stated otherwise, every script which we refer to from here on is implemented in {Python}.

\subsection{Data collection and parsing}
\label{subsec:scraping}

We have implemented a script for collecting the data. It scrapes the most popular sites\footnote{According to Alexa.} in the \texttt{.dk} domain. To make this as robust as possible, we inject a Firefox \texttt{User-agent} header into all requests, such as to fool the web-servers into giving useful responses\footnote{It is quite common for websites to reject crawlers. By injecting a \texttt{User-agent} header, a browser identifies itself to the webserver.}. Also, we make sure to deal with {HTTP} redirects whenever present. In spite of this, some sites fail to be downloaded correctly because of various {HTTP}/server errors. However, these problems does not occur in any predictable patterns, and should not contaminate our data set greatly. Whenever we receive a successful response, we store both the raw {HTML} and the {HTTP} response headers.

When the website data has been downloaded, we store and parse the {HTML} into a data structure which makes it easier to query. We have found that {Python}'s built-in \texttt{HTMLParser} is surprisingly fragile to the noisy and inconsistent nature of malformed {HTML}. Hence, various parser errors occur, resulting in a substantial data loss. Instead, we found that the {Beautiful Soup} library combined with the {lxml} {HTML}-parser is much more robust, and has helped us avoid data loss during this step \cite{beautiful_soup, lxml}.

\subsection{Preprocessing}
\label{subsec:preprocessing}

With the process described in \ref{subsec:scraping}, we can process the raw data into an object-oriented representation of the collected websites.

Apart from the raw website data itself, we enrich our data set with information from Alexa and Google on general web-statistics, as described in section \ref{subsec:attributes}. These numbers are easily accessible, and simply too relevant to be ignored. A complete list of the attributes and their meaning can be found in table \ref{tab:all_attributes} in Appendix \ref{apx:attributes}.

A high-level view of the architecture of the preprocessing library can be seen in figure \ref{fig:preprocess}. The \texttt{Website} class encapsulates the information belonging to each website. To easily extract the desired attributes, we introduce the concept of \textit{scanners}. Provided  a \texttt{Website} object, a scanner knows how to extract, and possibly bin/discretise, a given attribute from the raw website data. As such, there is a one-to-one correspondence between scanners and attributes. This scanner-based design is made pluggable, such that scanners can be added and removed as desired.

\hfig{figures/architecture.pdf}{0.7}{A high-level overview of the preprocessing library. Data is fetched from the internet, or potentially Statistico, and stored. The raw data is then then converted into \texttt{Website} objects which are processed using a set of scanners. The processed data is eventually stored in the {ARFF} file format.}{fig:preprocess}

Apart from generating the data set, we also perform a few extra preprocessing steps before we analyse the data. In order to secure consistency in any distance measures, numeric attributes are normalised into the domain $[0, 1]$. Further, irrelevant attributes are removed from the data set (please refer to section \ref{subsec:attributes}) and missing values are replaced by the mean (numeric attributes) or mode (nominal attributes).

Finally, the data set was saved in the ARFF format, which can be read by most data mining tools available.

\subsection{Datamining tools}
\label{subsec:weka}

For creating basic statistics and plots of the data, we have used the {Python} libraries {MatplotLib} and {Pandas} \cite{matplotlib, pandas}. The corresponding results will be discussed in section \ref{sec:statistics} and the full output can be found in appendix \ref{apx:basic_statistics}.

For the remaining data mining process we have used RapidMiner \cite{rapidminer}. This is as an alternative to Weka, that we think delivers a much better user experience and a more powerful workflow \cite{weka}. In RapidMiner one sets up a so called \textit{process} that consists of operators, and possibly \textit{building blocks} consisting of other operators. Our results are discussed in section \ref{sev:analysis}.

%\todo{Mangler der noet her? Jeg har udkommenteret en hel del i latex'en herefter...}
% In order to use both these tools, the preprocessing library (refer to section \ref{subsec:preprocessing}) was developed such that it could output the data set in the \texttt{ARFF} format, which can be read by both Weka and RapidMiner.

% Three versions of the data set were created, in order to support all potential use cases:

% \begin{itemize}
% \item A version containing the raw values from the data set.
% \item A version in which every value was discretised/binned.
% \item A combination of the former two, in which only a select few attributes were binned\footnote{These binned attributes were all nominal. In other words, we reduced the number of bins belonging to these attributes. For instance, we grouped all versions of the Apache webserver, instead of considering the specific versions.}.
% \end{itemize}

% It turned out that the combined version was easiest to work with, and was thus chosen for the actual data analysis.

% Although the data set is already been preprocessed, we performed a few extra preprocessing steps using RapidMiner. In order to secure consistency in any distance measures, the numeric attributes were normalised into the domain $[0, 1]$. Further, irrelevant attributes were removed from the data set (please refer to section \ref{subsec:attributes}).