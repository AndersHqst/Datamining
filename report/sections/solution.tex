\section{Solution}
\label{sec:solution}

We now present our solution, which primarily deals with collection and preprocessing of the data. We have implemented a small preprocessing library that uses a plugin architecture which allows us to add, remove, and alter the way in which we extract features.

\subsection{Preprocessing}
\label{subsec:preprocessing}

% Class-like diagram showing analyser, scanners, website, scraper, calls to alexa.

As mentioned in section \ref{subsec:scraping}, a script was created to download and scrape the websites of interest. The scraped websites (and auxiliary information) were stored on disk for easy access, using a simple and easy to parse data format.

Apart from the websites themselves, auxiliary information was fetched from Alexa and Google. Similar scripts were created to fetch and parse this data. Please refer to the actual source code for more information on this. Further, a list of the actual attributes in the data set can be found in table \ref{tab:all_attributes}.

A small library was developed to process the raw data (including \texttt{HTML}, \texttt{HTTP} headers, Alexa information, Google PageRank) into a finite set of well defined attributes, which could be used for subsequent data analysis.

A high-level view of the architecture of this preprocessor library can be seen in figure \ref{fig:preprocess}. A specific class (\texttt{Website}) was used to hold the information belonging to each website (including both the raw and parsed \texttt{HTML}). In order to easily extract the desired features, we introduced the concept of \textit{scanners}. Given a \texttt{Website} object, each scanner can extract and preprocess a specific feature from the raw data. This was implemented in a pluggable way, such that scanners could be added/removed as needed. One scanner is implemented for each attribute described in section \ref{subsec:attributes}.

\hfig{figures/architecture.pdf}{0.7}{A high-level overview of the preprocessing library. Data is fetched from the internet (or potentially Statistico) and stored in a specific data format. These data files are then converted into \texttt{Website} objects which are analysed using several different scanners. The result is saved as an \texttt{ARFF} file.}{fig:preprocess}

\subsection{Datamining tools}
\label{subsec:weka}

% How is data fed to weka. Problems?

We decided to use Weka and RapidMiner for the actual data analysis. Weka was used for the initial experimentation, but the richer feature set found in Rapidminer was used for the results presented here. 

In order to use both these tools, the preprocessing library (refer to section \ref{subsec:preprocessing}) was developed such that it could output the data set in the \texttt{ARFF} format, which can be read by both Weka and RapidMiner.

Three versions of the data set were created, in order to support all potential use cases:

\begin{itemize}
\item A version containing the raw values from the data set.
\item A version in which every value was discretised/binned.
\item A combination of the former two, in which only a select few attributes were binned\footnote{These binned attributes were all nominal. In other words, we reduced the number of bins belonging to these attributes. For instance, we grouped all versions of the Apache webserver, instead of considering the specific versions.}.
\end{itemize}

It turned out that the combined version was easiest to work with, and was thus chosen for the actual data analysis.

Although the data set had already been preprocessed, we performed a few extra preprocessing steps using RapidMiner. In order to secure consistency in any distance measures, the numeric attributes were normalised into the domain $[0, 1]$. Further, irrelevant attributes were removed from the data set (please refer to section \ref{subsec:attributes}).

\subsection{Attributes}
\label{subsec:attributes}

The complete list of attributes in the data set can be found in table \ref{tab:all_attributes} in appendix \ref{apx:attributes}. The attributes with the \texttt{alexa} prefix have been downloaded from Alexa and includes the language of the websites, the number of incoming links, the average load time and the global and Danish rank of the websites\footnote{The Alexa rank is an estimate of the actual visitor count rank of the website. For instance, a global Alexa rank of 1 is assigned the most visited website.}. Attributes with the \texttt{has\_content} prefix, indicate whether or not the website contains some specific content (such as technological content). The \texttt{has\_js} attributes indicate whether or not various JavaScript libraries are used. For a description of the remaining attributes, please refer to the table.

As mentioned earlier, irrelevant attributes were removed from the data set, in order to render an analysis possible. Hence, attributes which were more or less static across the whole dataset were removed, as these provided no further insight about the websites. The selection process was performed manually, as this made it possible to individually asses the importance of the different attributes.

Further, the reduced data set was grouped into three logical subsets, each accounting for some specific aspect of the websites. These final subsets can be seen in table \ref{tab:cluster_attributes}. These subsets were used for the actual analysis.

\htab{l l}
{
\toprule
Subset & Attributes in subset\\
\midrule
Rank subset & \texttt{alexa\_load\_time}, \texttt{alexa\_rank}, \texttt{alexa\_rank\_dk}, \\
& \texttt{alexa\_links\_in}, \texttt{internal\_links\_count}, \\
& \texttt{external\_links\_count}, \texttt{page\_rank}, \texttt{title\_tag}, \\
& \texttt{has\_description}, \texttt{has\_keywords}, \texttt{img\_count} \\
\midrule
Technology subset & \texttt{html5}, \texttt{html5\_tags}, \texttt{has\_js\_jquery}, \\
& \texttt{server}, \texttt{cms}, \texttt{has\_analytics} \\
\midrule
Content subset & \texttt{has\_content\_business}, \texttt{has\_content\_film}, \\
& \texttt{has\_content\_food}, \texttt{has\_content\_games}, \\
& \texttt{has\_content\_health}, \texttt{has\_content\_music}, \\
& \texttt{has\_content\_news}, \texttt{has\_content\_shop}, \\
& \texttt{has\_content\_sport}, \texttt{has\_content\_technology}, \\
& \texttt{has\_content\_transport}, \texttt{has\_content\_xxx} \\
\bottomrule
}{The different subsets used for in clustering process.}{tab:cluster_attributes}