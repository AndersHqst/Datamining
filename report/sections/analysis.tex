\section{Analysis}
\label{sec:analysis}

In this section we present our analysis of the obtained data set. First we discuss a few results from our basic statistics analysis, that we interesting. Then we describe the outcome of looking for clusters in the data. And finally we describe how we tried to predict a websites \texttt{PageRank}.

\subsection{Statistics}
\label{subsec:statistics}
To get a better understanding of the data, we have automated to process of generating some basic statistics and overview of the data. This has helped us in getting a general understanding of the data, and consequently removing certain attributes from the data as described in Section \ref{subsec:attributes} We plot every attribute on histograms or bar charts, along the minimum, maximum, mode, standard deviation, and median for numerical values. All non binary attributes are plotted in bins of 10\% intervals of the data, and top and bottom 5 values are listed along the plot. Numerical attributes are also plotted in their inter-quantile range, as outliers often give a skewed picture. All plots can be found in Appendix \ref{apn:basic_statistics}. We found that the JavaScript libraries we look for are very uncommon, except JQuery which is on 1 out of 3 sites. Servers seem to be distribution much as we would expect, and we are often able to identified these. Servers binned to ignore version numbers is show in figure \ref{fig:servers_top_10_discrete}.

\hfig{figures/basic_statistics/servers_top_10_discrete.png}{0.7}{Servers Top 10 Discrete}{fig:servers_top_10_discrete}

It turns out that is harder to identify the CMS used by a given site, but as can be seen in figure \ref{fig:cms_distribution_binned_discrete} we can get a distribution that seems very likely. The most doubtful outcome is Drupal, that is known to be very popular, but is often harder to detect by the scanners.

\hfig{figures/basic_statistics/cms_distribution_binned_discrete.png}{0.65}{Cms Distribution Binned}{fig:cms_distribution_binned_discrete}

If we look at the \texttt{external\_links\_count} attribute, that is, links from the website pointing to another site, figure \ref{fig:external_links_count} and \ref{fig:external_links_count_interquartile} show, that most sites have less than 106 external links, and the the descending in the number of links is clear in figure \ref{fig:external_links_count_interquartile}

%\hfig{figures/basic_statistics/external_links_count.png}{0.8}{External Links Count}{fig:external_links_count}

\hfig{figures/basic_statistics/external_links_count_interquartile.png}{0.8}{External Links Count Interquartile}{fig:external_links_count_interquartile}

An interesting and important stat is the number of back-links to a given website. Table \ref{tab:alexa_links_in_back_links} shows that on the top 10 of sites with most back links, 5 sites are from educational institutions, with ku.dk surpassing even google.dk. A complete list of top and bottom 10 of the numerical attributes can be found in Appendix \ref{apx:tables}.

\hhtab{p{130pt}p{50pt}}
{
\toprule
Site & Links\\
\midrule
ku.dk & 13964\\
google.dk & 11826\\
blogspot.dk & 10921\\
dr.dk & 8617\\
cmsimple.dk & 8369\\
au.dk & 7557\\
ots.dk & 5968\\
ucsj.dk & 5886\\
aau.dk & 5882\\
smartlog.dk & 5558\\
\bottomrule
}{Danish sites with most back-links according to Alexa. Back-links are often considered important for good page rank}{tab:alexa_links_in_back_links}

We find that most of the numbers we get from collecting these statistics, are much as we would imagine, but we also learned that some attributes are hard to detect. However, subject to our own gut-feeling, the distributions seem representative, and thus gives us some confidence that we can perform the subsequent data mining process with truthful, or at least convincing, results. Note, that it is not possible to actually validate or compare our numbers. No one can know the what actual distribution is of most of our attributes.


\subsection{Clustering}
\label{subsec:clustering}

The data was clustered using \texttt{K-Means}, \texttt{K-Medoids}, and \texttt{DBSCAN}.

We have experimented with \texttt{K-Means} and \texttt{K-Medoids} to look for 8 clusters\footnote{The choice of 8 clusters were based on experimentation and seemed to yield the most well-defined clusters.}, and we use a mixed distance measure as described in the course book \cite{book}. For \texttt{DBSCAN} we have sought for a minimum of 5 points using $\varepsilon = 1.0$.

\todo{Lidt uklart hvad vi vil sige med det her afsnit. Synes det skal forkortes til kun at sige at vi havde mest succes med k-medoids, eller give et svar på hvorfor de andre fejler og hvad vi kan bruge det til/hvad det fortæller os}
Most of our clustering analysis is based on the \texttt{K-Medoids} clusters, as we experienced a few problems using the remaining two algorithms. Somehow, the \texttt{K-Means} implementation in \texttt{RapidMiner} returned less than 8 clusters. However, these clusters still seemed to be in good correspondence with the \texttt{K-Medoids} clusters, although somewhat less fine-grained. In some cases, using \texttt{DBSCAN} resulted in a very large number of clusters. As the data set has a lot of binary attributes, changing the value of $\varepsilon$ does not have much effect, as the usual result is a very rapid transitions of the number of clusters. Still, in the cases where \texttt{DBSCAN} did not produce an excessive amount of clusters, the clustering seemed to be in quite good agreement with the clusters we found using \texttt{K-Medoids}.

The clustering has been done for each subset of the attributes listed in Table \ref{tab:cluster_attributes}.

\subsubsection{Clustering the rank subset}

For the rank subset attributes, all three clustering algorithms can find significant clusters, but here we will just describe the results of \texttt{K-Medoids}. Below we have named the clusters found along a description of what the cluster tells us.

\paragraph{Cluster 1.1: The search-engine optimized top websites}

This cluster has the best average \texttt{Alexa} rank and a very good \texttt{PageRank}. These websites contain both the keyword and description meta tags. The fact that time has been spent on these\texttt{SEO} related attributes, could indicate that this cluster represents websites which use the web as their main platform. This cluster contains sites such as \verb|www.krak.dk|, \verb|www.dba.dk| og \verb|www.amino.dk|.

\paragraph{Cluster 1.2: The unoptimized top websites}

This cluster has a very good \texttt{Alexa Rank} and \texttt{PageRank}. However, less time has been spent on \texttt{SEO}, which indicates that these websites are the ones that have become popular through some other media or do not consider this optimization as much as the former cluster. This cluster includes \verb|www.google.dk|, \verb|www.ekstrabladet.dk| and \verb|www.bt.dk|. Obviously, SEO is less important for Google itself.

\paragraph{Other observations} Other clusters can found for the less popular sites, but these were not as distinctly defined as the above mentioned clusters.

\subsubsection{Clustering the technology subset}

Clustering on the technology related attributes also showed some very distinct clusters. The most interesting clusters are discussed below. Again, the discussion is based on the results of K-Medoids.

\paragraph{Cluster 2.1: High-tech websites}

This cluster primarily consists of the websites which use both \texttt{HTML5} and the \texttt{JQuery JavaScript} library. Although \texttt{HTML5} and JavaScript libraries such as \texttt{JQuery} are becoming more and more common, these websites are the ones that have readily embraced these technologies.

\paragraph{Cluster 2.2: Websites using a CMS}

A specific CMS cluster was found. Although, the \texttt{CMS} usage data in our data set is somewhat underestimated, a very clear grouping of \texttt{CMS} systems was found. \texttt{Wordpress}, \texttt{Sharepoint} and \texttt{Drupal} websites are the most dominant \texttt{CMS}'s in this clusters. Further, \texttt{Google Analytics} is used in most of these websites.

\paragraph{Cluster 2.3: Low-tech websites}

Finally, we have the websites which do not use either \texttt{HTML5} or \texttt{JQuery}. |texttt{K-Medoids} actually grouped these websites into two different clusters --- a cluster containing low-tech sites running on \texttt{Microsoft} servers and a cluster containing the remaining low-tech sites. The major difference between these is the fact that the \texttt{Microsoft} low-tech cluster still makes use of \texttt{Google Analytics}, whereas this is not the case for the other cluster.

\paragraph{Other observations}

One interesting observation is the fact that the use of \texttt{Google Analytics} does not seem to be correlated with the general use of technology on the websites. However, this might be due to the fact, that using \texttt{Google Analytics} or similar services, is often a management decision, whereas management might be somewhat oblivious to the choice technology.

\subsubsection{Clustering the content subset}

Clustering on the content results in a few prominent clusters. The most extreme of these clusters are websites without any content\footnote{In this context, a website not having any content, really means that none the content scanners could categorize the website.} and the websites which satisfied almost all of our content criteria. For this clustering, we have noted that the content rich websites have a better \texttt{Google PageRank}.

Based on this observation, it makes sense to compare the amount of content in each clusters to its assigned \texttt{PageRank}. We make this comparison by plotting the average \texttt{PageRank} of each clusters versus its average of the content attributes\footnote{Although these attributes are actually binary, they were treated as numeric, as the average is more useful than the mode in this particular case.}.

As it turns out, there seem to be a relationship between the average amount of content in the cluster and the assigned \texttt{PageRank}. This tendency can be seen in figure \ref{fig:cluster_content} for \texttt{K-Medoids} and in appendix \ref{app:content_pagerank} for \texttt{K-Means} and texttt{DBSCAN}. Obviously, it must be noted that the exact relationship is still somewhat unclear, but the tendency is definitely present.

\hfig{figures/kmedoids_content_pagerank.png}{0.5}{The content attributes clustered using K-Medoids. It is seen, that a large amount of content tends to result in a better PageRank.}{fig:cluster_content}

\todo{Jeg synes afsnittet bør stoppe her (udkommenteret text i latex'en)- det er en bedre konklusion/afslutning der kommer lige før..}
% One could also plot the unclustered content attributes against the \texttt{PageRank}. This turned out to give a much more noisy picture, although the same tendencies were found on average. If course, it should also be noted that the clustering does not really provide any insight which could not be found from the original dataset. However, the clustering helped us \textit{spot} the tendency.

\subsection{Predicting Page Rank}
\label{subsec:predection}
We want to see if we can predict the page rank of a website, and for this we have set up a process that models linear and polynomial regression, and a neural network\footnote{Appendix \ref{apn:rapidminer_setup} shows examples of the RapidMinerGUI for setting up a data mining process.}. For this process we have used the mixed data set, and replaced missing values with the average of a given attribute. The data set has a total of 2332 missing values with is 2.18\% of the entire set, so we would expect these to have little to no influence on the generalization of our results. Table \ref{tab:prediction_results} lists the results of running predictions on both the Rank subset related subset, and the full set.

\todo{Consider delete of the comment column.}
\hhtab{p{80pt}p{50pt}p{50pt}p{50pt}}
{
\toprule
Model & Data & Error \\
\midrule
Linear Regression & Rank subset &  1.2541\\
Linear Regression & Content subset & 1.6248 \\
Linear Regression & Technology subset & 1.5842\\
Linear Regression & Full set & 1.4730\\
Neural Net & Rank subset & 1.6955 \\
Neural Net & Technology subset & 1.6498\\
Neural Net & Content subset & 1.7424 \\
Neural Net & Full set & 2.141 \\
\bottomrule
}{Overview of model predicting with Page Rank as the target value. The Error is the root mean squared error (RMSE) after cross validation. Target value is in its original scale.}{tab:prediction_results}

Page rank predictions are based on the cross-validation building-block in RapidMiner, which provides a few strategies like {\it leave-one-out} and and some randomized approaches. As our data is already in a random order we have simply used a linear sampling that separates a standard 10\% as test data, and thus performs 10 runs. As table \ref{tab:prediction_results} indicates we seem to be able to predict the page rank within \(\sqrt{1.573} = 1.2541\) which is also what we would expect, as the Alexa Rank and links related attributes are provided. The above predictions have also been tested with RapidMiner's build in M5 for automatic feature selection, but the results were virtually the same. It seems only a few of the JavaScript related features are excluded.
The output of the neural net model indicates that the Alexa links in attribute is the most important in deciding the page rank. This can be seen by inspecting the importance, or strength, that each attributes has on the outcome of the neural net. The visualization is show in figure \ref{fig:neural_net}. It is a bit surprising that the the neural net cannot produce a result closer to the linear regression linear regression. We have used parameter optimization in RapidMiner and experimented with 50, 500, and 5000 epocs, and the best resutl is as listed in table \ref{tab:prediction_results}

\dbgfig{../figures/neural_nets.png}{0.9}{Importance of a connection between neurones in the neural net are indicated by the boldness of the connection. The Alexa links in is in the input layer 7th from the top, the bias is at the bottom.

The correlation coefficient between page rank and and Alexa links in is 0.2054 with a p value of 0.1808 which supports the above indications. As this is somewhat an undesired conclusion, the above models for prediction are test against a sub set of the Rank subset minus attributes from Alexa as seen in table \ref{tab:rank_no_alexa}, which yields the results seen in table \ref{tab:rank_no_alexa_prediction}

\hhtab{l l}
{
\toprule
Subset & Attributes in subset\\
\midrule
Rank subset & \texttt{external\_links\_count}, \texttt{page\_rank}, \texttt{title\_tag}, \\
& \texttt{internal\_links\_count}, \texttt{has\_description}, \\
&\texttt{has\_keywords}, \texttt{img\_count} \\
\bottomrule
}{Rank subset without Alexa attributes.}{tab:rank_no_alexa}

\hhtab{p{80pt}p{50pt}p{50pt}p{50pt}}
{
\toprule
Model & Data & Error & Comment\\
\midrule
Linear Regression & Rank subset set & 2.808 & -  \\
Neural Net & Rank subset set & 2.924 & - Learning rate: 0.3, momentum: 0.2, 500 epocs\\
\bottomrule
}{Overview of model predicting with Page Rank as the target value. The Error is the average squared error after cross validation. Target value is in its original scale.}{tab:rank_no_alexa_prediction}

As table \ref{tab:rank_no_alexa_prediction} shows, error has gone up by removing the Alexa related attributes, and shows that we can predict a page rank in the \(\sqrt{2.875} = 1.6955\) range. The interesting thing to say about this is, that its actually not a bad prediction considering the scarce information used. We guess that the connection here is simply, that sites that rank well on Google also manage their websites well, and thus have the above attributes.

As a final remark, we do not run any classifiers. We could have tried seeing the \texttt{PageRank} as different classes and benchmark the accuracy of a neural network, a support vector machin (SVM), or some other model, but we argue that regression makes most sense in this case, as \texttt{PageRank}