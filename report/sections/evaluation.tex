\section{Analysis}
\label{sec:analysis}

% \todo{I imagine this is our prime section.}
% Diagrams and numbers. Evaluate and reflect upon all results. Can we say anything?

Obviously, generating the data set is not that interesting in itself. Hence, basic statistical data was extracted, such as to provide a general overview of the most interesting information, including popular websites, choices of CMS and so forth. 

Further, clustering algorithms were used to try and identify any obvious clusters in the data. Classification/prediction algorithms were also tested, but proved less useful due to the noisy nature of the data. 

\subsection{Statistics}
\label{subsec:statistics}

Some statistics here please :)

\subsection{Clustering}
\label{subsec:clustering}

The data was clustered using three different clustering algorithms:

\begin{itemize}
\item K-Means
\item K-Medoids
\item DBSCAN
\end{itemize}

K-Means and K-Medoids were configured to look for 8 clusters\footnote{The choice of 8 clusters were based on experimentation and seemed to yield the most defined clusters.}. A mixed distance measure was used, similar to the one described in the course book \cite{book}. DBSCAN was configured to look for a minimum of 5 points using $\varepsilon = 1.0$.

Three different subsets of attributes were used for the clustering. One subset was chosen to represent the attributes that might influence the ranking of the websites. A second subset focused on the technological attributes. Finally, one subset was created to represent the content of the websites. The specific attributes can be seen in table \ref{tab:cluster_attributes}.

\hhtab{l l}
{
\toprule
Subset & Attributes in subset\\
\midrule
Rank subset & \texttt{alexa\_load\_time}, \texttt{alexa\_rank}, \texttt{alexa\_rank\_dk}, \\ 
& \texttt{alexa\_links\_in}, \texttt{internal\_links\_count}, \\ 
& \texttt{external\_links\_count}, \texttt{page\_rank}, \texttt{title\_tag}, \\ 
& \texttt{has\_description}, \texttt{has\_keywords}, \texttt{img\_count} \\
\midrule
Technology subset & \texttt{html5}, \texttt{html5\_tags}, \texttt{has\_js\_jquery}, \\
& \texttt{server}, \texttt{cms}, \texttt{has\_analytics} \\
\midrule
Content subset & \texttt{has\_content\_business}, \texttt{has\_content\_film}, \\ 
& \texttt{has\_content\_food}, \texttt{has\_content\_games}, \\ 
& \texttt{has\_content\_health}, \texttt{has\_content\_music}, \\
& \texttt{has\_content\_news}, \texttt{has\_content\_shop}, \\ 
& \texttt{has\_content\_sport}, \texttt{has\_content\_technology}, \\
& \texttt{has\_content\_transport}, \texttt{has\_content\_xxx} \\
\bottomrule
}{The different subsets used for in clustering process.}{tab:cluster_attributes}

\subsection{Prediction and classification}
\label{subsec:predictclassify}

This does not seem to work. Why? Loosely correlated features? Why does K-Means clustering make sense when K-NN does not???