\section{Introduction}
\label{sec:introduction}

This report describes a data mining project, inspired by novel business concept developed by Statistico in Aarhus Denmark. Statistico is a subsidiary of a company that specializes in Search Engine Optimization (SEO) and web development. Statistico has build a system that scrapes HTML from every website in the .dk domain, one link deep, once a month. The data is stored it in a Lucene\footnote{Lucene is under Apace and is a widely for building indices. http://lucene.apache.org/core/} index, with Solr\footnote{Solr is a server for sending text based queries to the Lucene index. http://lucene.apache.org/solr/} build on top for querying the data. The idea is to make website meta information and statistics available for web-based businesses. Examples could be to get information on the current distribution of CMS, or how widely a new JavaScript library has been adopted. A total of 600 GB of data from around 2.5 million pages are scraped every month. Currently their work is focused on optimizing the data collection, data integrity, search and storage.
The future challenge is to extract actual value of the data. This is larger task that will require work on data visualization and non the least data mining. Currently the company has several visions for what kind of information they want to make available, but no real empirical knowledge or experience with mining the data. As such, this project can be seen as a pilot project for performing data mining on the kind of website meta data that the Statistico wants to build its business on.

\paragraph{Data set}
\label{subsec:data_set}
With the aforementioned size of the data set in mind, we have decided to only work on a subset of the data. Therefore our solution includes a component that scrapes HTML, HTTP-headers, and robot files from 2425 websites in the .dk domain, which we expect to be enough for our results to generalize quite well. To extract the features from HTML data that we are interested in, a substantial amount of preprocessing has to be done, and as will be explained in the next section, we have also chosen work around the indexing technologies used by Statistico. However, it should be clear that our solution can readily by used on the Statistico data set, with the constraint that it would take one machine a few weeks to perform the preprocessing.

\paragraph{Motivation}
In this project we have been highly motivated by the fact that we have had the chance to work on a data mining project that is relevant for a real-world company. We thank Statistico for their time involving us in their work, and the resources and time they have invested in getting the project underway.

\paragraph{Overview}
In the next section we formalize the problem at hand, and discuss our approach to the data mining process. In section \ref{sec:solution} the solution, and in particular, the preprocessing is described. In section \ref{sec:evaluation} we evaluate and report on our results. Finally in section \ref{sec:conclusion} we summarize and conclude.